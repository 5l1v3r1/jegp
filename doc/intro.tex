%!TEX root = std.tex
\rSec0[scope]{Scope}

\pnum
This document describes the contents of the \defn{JEGP library}.

\rSec0[refs]{References}

\pnum
The following documents are referred to in the text
in such a way that some or all of their content
constitutes requirements of this document.

\begin{itemize}
\item ISO/IEC 14882:2020, \doccite{Programming Languages --- C++}
\end{itemize}

\pnum
ISO/IEC 14882 is herein called the \defn{C++ Standard}.

\rSec0[intro]{Introduction}

\rSec1[intro.general]{General}

\pnum
The library specification subsumes the C++ Standard's [library],
assumingly amended to the context of this library.
\begin{example}
\begin{itemize}
\item Per C++ Standard's [namespace.future], \tcode{::jegp2} is reserved.
\item Per C++ Standard's [contents]\#3, a name \tcode{x} means \tcode{::jegp::x}.
\end{itemize}
\end{example}
The following subclauses describe additions to it.

\begin{floattable}{Library categories}{library.categories}
{ll}
\topline
\hdstyle{Clause}        & \hdstyle{Category}          \\ \capsep
\ref{utilities}         & General utilities library   \\
\end{floattable}

\rSec1[requirements]{Library-wide requirements}

\rSec2[contents]{Library contents}

\pnum
Whenever a name is qualified with \tcode{X::}, \tcode{::X::} is meant.
\begin{example}
When \tcode{std::Y} is mentioned, \tcode{::std::Y} is meant.
\end{example}

\rSec2[reserved.names]{Reserved names}

\pnum
The JEGP library reserves macro names starting with \tcode{JEGP_}.
